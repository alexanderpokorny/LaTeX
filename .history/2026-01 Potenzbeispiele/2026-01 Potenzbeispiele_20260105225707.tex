\documentclass[a4paper,11pt]{scrartcl}
\usepackage[utf8]{inputenc}
\usepackage[T1]{fontenc}
\usepackage[ngerman]{babel}
\usepackage{amsmath, amssymb}
\usepackage{geometry, wasysym}
\usepackage{enumitem}
\usepackage{fancyhdr}
\usepackage{multicol}
\usepackage{tikz}
\usepackage{eurosym}
\usepackage{booktabs}
\usepackage{tabularx}

% Seiteneinstellungen
\geometry{left=2.5cm, right=2.5cm, top=2.5cm, bottom=2.5cm}
\setlength{\parindent}{0pt}
\setlength{\parskip}{6pt}

% Kopf- und Fußzeile
\pagestyle{fancy}
\fancyhf{}
\lhead{\textbf{Mathematik Übungspool}}
\rhead{3. Klasse Gymnasium - 2. Schularbeit}
\cfoot{\thepage}

\begin{document}

\section{Großes Potenzen-Training (Intensiv)}\vspace{1cm}

\subsection{Warm-up}

\textbf{Aufgabe 23 (Null und Eins):}
Vereinfache sofort im Kopf.
\begin{multicols}{2}
\begin{itemize}
    \item[a)] $x^1 = $ \underline{\hspace{2cm}}
    \item[b)] $5^0 = $ \underline{\hspace{2cm}}
    \item[c)] $(3a)^0 = $ \underline{\hspace{2cm}}
    \item[d)] $-7^0 = $ \underline{\hspace{2cm}}
    \item[e)] $(-7)^0 = $ \underline{\hspace{2cm}}
    \item[f)] $a^0 + a^1 = $ \underline{\hspace{2cm}}
\end{itemize}
\end{multicols}

\textbf{Aufgabe 24 (Vorzeichen-Lotto):}
Berechne den Wert. Achte genau darauf, ob die Basis negativ ist oder das Minuszeichen nur davor steht.
\begin{multicols}{2}
\begin{itemize}
    \item[a)] $(-2)^3 =$
    \item[b)] $-2^3 =$
    \item[c)] $(-2)^4 =$
    \item[d)] $-2^4 =$
    \item[e)] $(-1)^{10} =$
    \item[f)] $(-1)^{11} =$
\end{itemize}
\end{multicols}

\subsection{Level 1: Rechnen mit Basis und Exponent}

\textbf{Aufgabe 25 (Produkte und Quotienten):}
Fasse zu einer Potenz zusammen.
\begin{itemize}
    \item[a)] $m^4 \cdot m^3 \cdot m = $
    \item[b)] $x^2 \cdot x^5 \cdot x^{-3} = $
    \item[c)] $\dfrac{a^{10}}{a^2} = $
    \item[d)] $\dfrac{y^5}{y^5} = $
    \item[e)] $10^7 : 10^{-2} = $
\end{itemize}

\textbf{Aufgabe 26 (Potenzieren von Potenzen):}
Löse die Klammern auf. Regel: $(x^a)^b = x^{a \cdot b}$.
\begin{itemize}
    \item[a)] $(a^4)^2 =$
    \item[b)] $(y^{-2})^3 =$
    \item[c)] $(k^3)^{-4} =$
    \item[d)] $((x^2)^3)^2 =$
\end{itemize}

\subsection{Level 2: Klammern mit Koeffizienten und Brüchen}

Hier passieren die meisten Fehler. Vergiss nicht: Die Zahl (der Koeffizient) wird ganz normal potenziert, die Hochzahlen der Variablen werden multipliziert.

\textbf{Aufgabe 27 (Klammern auflösen):}
\begin{itemize}
    \item[a)] $(2x)^3 =$
    \item[b)] $(3a^2)^2 =$
    \item[c)] $(5x^3 y^2)^2 =$
    \item[d)] $(-2a^3)^3 =$
    \item[e)] $(-3x^2)^2 =$
    \item[f)] $\left(\dfrac{x}{2}\right)^3 =$
    \item[g)] $\left(\dfrac{2a^2}{3b^3}\right)^2 =$
\end{itemize}

\subsection{Level 3: Komplexe Vereinfachungen}

Bringe die Terme auf die einfachste Form. Das Ergebnis soll keine negativen Exponenten und keine Doppelbrüche enthalten.

\textbf{Aufgabe 28 (Multiplikation und Division gemischt):}
\begin{itemize}
    \item[a)] $3x^2 \cdot 4x^3 \cdot 2x = $
    \item[b)] $\dfrac{15a^5 \cdot 2a^3}{6a^4} = $
    \item[c)] $\dfrac{(2x)^3 \cdot x^2}{4x^4} = $ \quad \textit{(Zuerst Klammer auflösen!)}
    \item[d)] $\dfrac{12x^5 y^3}{3x^2 y^4} = $ \quad \textit{(Kürze soweit wie möglich)}
\end{itemize}

\textbf{Aufgabe 29 (Negative Exponenten):}
Schreibe ohne negative Hochzahlen (als Bruch).
\begin{itemize}
    \item[a)] $x^{-3} =$
    \item[b)] $2a^{-2} =$
    \item[c)] $(2a)^{-2} =$
    \item[d)] $\dfrac{1}{x^{-4}} =$
    \item[e)] $\left(\dfrac{a}{b}\right)^{-1} =$
\end{itemize}

\textbf{Aufgabe 30 (Große Term-Vereinfachung):}
Vereinfache schrittweise.
\[ \frac{5a^3 \cdot (2a^2)^3}{10 \cdot (a^2)^4} \]
\begin{enumerate}
    \item Klammern im Zähler und Nenner auflösen.
    \item Zähler zusammenfassen.
    \item Kürzen.
\end{enumerate}
Ergebnis: \underline{\hspace{4cm}}

\subsection{Level 4: Wissenschaftliche Schreibweise (Gleitkomma)}

\textbf{Aufgabe 31 (Umwandeln):}
Vervollständige die Tabelle.
\begin{center}
\renewcommand{\arraystretch}{1.5}
\begin{tabular}{|l|c|c|}
\hline
\textbf{Zahl (Dezimal)} & \textbf{Wissenschaftlich ($m \cdot 10^k$)} & \textbf{Computer (E-Notation)} \\ \hline
$123\,000$ & $1,23 \cdot 10^5$ & $1,23\text{E}5$ \\ \hline
$0,0045$ & & \\ \hline
$7\,050\,000\,000$ & & $7,05\text{E}9$ \\ \hline
& $3,0 \cdot 10^{-4}$ & \\ \hline
$0,000\,000\,12$ & & \\ \hline
\end{tabular}
\end{center}

\textbf{Aufgabe 32 (Rechnen mit Zehnerpotenzen):}
\begin{itemize}
    \item[a)] $2,5 \cdot 10^4 \cdot 2 \cdot 10^3 = $ \quad \textit{(Zahlen multiplizieren, Zehnerpotenzen addieren)}
    \item[b)] $\dfrac{8 \cdot 10^9}{4 \cdot 10^6} = $
    \item[c)] Das Licht legt in einem Jahr ca. $9,46 \cdot 10^{12}$ km zurück. Der Durchmesser der Milchstraße beträgt ca. $100\,000$ Lichtjahre.
    Berechne den Durchmesser der Milchstraße in Kilometern und gib das Ergebnis in Gleitkommadarstellung an.
\end{itemize}

\subsection{Multiple Choice \& Verständnis}

\textbf{Aufgabe 33 (Finde die Fehler):}
Welche Rechnungen sind \textbf{falsch}? Kreuze an (es können mehrere sein).
\begin{itemize}
    \item[$\square$] $x^2 + x^3 = x^5$
    \item[$\square$] $x^2 \cdot x^3 = x^5$
    \item[$\square$] $(x^2)^3 = x^5$
    \item[$\square$] $5 \cdot 10^{-2} = -500$
    \item[$\square$] $\frac{x^5}{x^{-2}} = x^7$
\end{itemize}

\textbf{Aufgabe 34 (Zuordnen):}
Verbinde die Terme, die das Gleiche bedeuten mit Linien.

\begin{center}
\begin{tabular}{rcl}
$2^{-2}$ & \hspace{2cm} & $-4$ \\
$(-2)^2$ & & $0,25$ \\
$-2^2$ & & $4$ \\
$2 \cdot 10^{-1}$ & & $0,2$ \\
\end{tabular}
\end{center}

\section{Das große Finale: Komplexe Termketten}

\textbf{Aufgabe 35 (Der „Endgegner“):}
Vereinfache die folgenden Terme so weit wie möglich.
\textit{Tipp:} Halte dich streng an die Reihenfolge:
\begin{enumerate}
    \item Klammern auflösen (Potenzieren).
    \item Punktrechnungen durchführen (Multiplizieren/Dividieren).
    \item Strichrechnungen durchführen (Nur Gleiches addieren/subtrahieren!).
\end{enumerate}

\begin{itemize}
    \item[a)] \textbf{Gemischte Basen (Addieren):} \\
    $3x^2 + 4y^2 - 2x^2 + 5y^2 + x^2 - 8y^2 + x^3 + 2x^3$
    
    \item[b)] \textbf{Multiplikationskette mit Vorzeichen:} \\
    $2a \cdot (-3a) \cdot a^2 \cdot (-2) \cdot a^0 \cdot (-a)^2 \cdot 5$
    
    \item[c)] \textbf{Klammern und Potenzen (Punkt vor Strich):} \\
    $(2x)^3 + 3 \cdot x^3 - (-2x^3) + x \cdot x^2 + (4x^2) \cdot x - x^0$
    
    \item[d)] \textbf{Brüche und Kürzen:} \\
    $\frac{4x^5}{2x^2} + \frac{9x^4}{3x} - \frac{10x^6}{5x^3} + x^2 \cdot x + \frac{x^7}{x^4}$
    
    \item[e)] \textbf{Negative Exponenten und Brüche:} \\
    $2x^{-2} \cdot x^5 + \frac{x^6}{x^3} - (x^2)^2 \cdot x^{-1} + 3x^3 - \frac{1}{x^{-3}}$
    
    \item[f)] \textbf{Mehrere Variablen (x und y):} \\
    $3x^2y + 4xy^2 - 2x \cdot (xy) + 5y \cdot (xy) - x^2y + 2xy^2$
    
    \item[g)] \textbf{Potenzieren von Potenzen (Kettenreaktion):} \\
    $(a^2)^3 \cdot a^4 \cdot (a^3)^2 : a^5 \cdot (a^2)^0 \cdot a$
    
    \item[h)] \textbf{Die "Null" und das Minus (Achtung Falle!):} \\
    $5x^0 + (5x)^0 - 5^0 + (-5)^0 + 5x - (-5x) + (-2)^2 - 2^2$
    
    \item[i)] \textbf{Versteckte Vereinfachung:} \\
    $\frac{(3a)^3}{9a} - 2a \cdot a + (2a)^2 - a^2 + \frac{10a^5}{2a^3}$
    
    \item[j)] \textbf{Der lange Bruch-Mix:} \\
    $\frac{4a^3b^2}{2ab} + 3a^2b - 2ab \cdot a + \frac{(ab)^3}{ab^2} - 5a^2b + a \cdot b$
\end{itemize}

\newpage
\subsection{Lösungen zum Potenzen-Training}

\textbf{Zu Aufgabe 23:}
a) $x$ \quad b) $1$ \quad c) $1$ \quad d) $-1$ (Minus steht davor!) \quad e) $1$ \quad f) $1+a$

\textbf{Zu Aufgabe 24:}
a) $-8$ \quad b) $-8$ \quad c) $+16$ \quad d) $-16$ \quad e) $+1$ \quad f) $-1$

\textbf{Zu Aufgabe 25:}
a) $m^{4+3+1} = m^8$ \quad b) $x^{2+5-3} = x^4$ \quad c) $a^{10-2} = a^8$ \quad d) $y^0 = 1$ \quad e) $10^{7-(-2)} = 10^9$

\textbf{Zu Aufgabe 26:}
a) $a^8$ \quad b) $y^{-6}$ oder $\frac{1}{y^6}$ \quad c) $k^{-12}$ \quad d) $x^{12}$

\textbf{Zu Aufgabe 27:}
a) $8x^3$ \quad b) $9a^4$ \quad c) $25x^6y^4$ \quad d) $-8a^9$ \quad e) $9x^4$ (Minus wird plus durch Quadrat) \quad f) $\frac{x^3}{8}$ \quad g) $\frac{4a^4}{9b^6}$

\textbf{Zu Aufgabe 28:}
a) $24x^6$ ($3 \cdot 4 \cdot 2 = 24$; $2+3+1 = 6$) \\
b) $\frac{30a^8}{6a^4} = 5a^4$ \\
c) Zähler: $8x^3 \cdot x^2 = 8x^5$. Bruch: $\frac{8x^5}{4x^4} = 2x$ \\
d) $\frac{4x^3}{y}$ (oder $4x^3y^{-1}$)

\textbf{Zu Aufgabe 29:}
a) $\frac{1}{x^3}$ \quad b) $\frac{2}{a^2}$ \quad c) $\frac{1}{4a^2}$ \quad d) $x^4$ \quad e) $\frac{b}{a}$

\textbf{Zu Aufgabe 30:}
1. Zähler: $(2a^2)^3 = 8a^6$. Nenner: $(a^2)^4 = a^8$. \\
2. Term: $\frac{5a^3 \cdot 8a^6}{10a^8} = \frac{40a^9}{10a^8}$. \\
3. Kürzen: $4a$.

\textbf{Zu Aufgabe 31:}
$4,5 \cdot 10^{-3}$ (4,5E-3) \\
$7,05 \cdot 10^9$ \\
$0,0003$ (3,0E-4) \\
$1,2 \cdot 10^{-7}$ (1,2E-7)

\textbf{Zu Aufgabe 32:}
a) $5 \cdot 10^7$ \\
b) $2 \cdot 10^3 = 2000$ \\
c) $100\,000 = 10^5$. Rechnung: $9,46 \cdot 10^{12} \cdot 10^5 = 9,46 \cdot 10^{17}$ km.

\textbf{Zu Aufgabe 33 (Falsch sind):}
\begin{itemize}
    \item $x^2 + x^3 = x^5$ (Addieren geht nicht!)
    \item $(x^2)^3 = x^5$ (Richtig wäre $x^6$, da multipliziert wird)
    \item $5 \cdot 10^{-2} = -500$ (Richtig wäre $0,05$)
\end{itemize}

\textbf{Zu Aufgabe 34:}
$2^{-2} \rightarrow 0,25$ \\
$(-2)^2 \rightarrow 4$ \\
$-2^2 \rightarrow -4$ \\
$2 \cdot 10^{-1} \rightarrow 0,2$


\textbf{Zu Aufgabe 35:}
\textbf{a)} Wir ordnen nach Sorten:
$(3x^2 - 2x^2 + x^2) + (x^3 + 2x^3) + (4y^2 + 5y^2 - 8y^2)$ \\
Ergebnis: $\mathbf{2x^2 + 3x^3 + y^2}$ (Achtung: $x^2$ und $x^3$ darf man nicht addieren!)

\textbf{b)} Zahlen: $2 \cdot (-3) \cdot (-2) \cdot 5 = 60$. \\
Variablen: $a \cdot a \cdot a^2 \cdot a^0 \cdot a^2$ (Hinweis: $(-a)^2 = a^2$). \\
Exponenten addieren: $1 + 1 + 2 + 0 + 2 = 6$. \\
Ergebnis: $\mathbf{60a^6}$

\textbf{c)}
1. Klammern/Produkte: $(2x)^3 = 8x^3$. \quad $3 \cdot x^3 = 3x^3$. \quad $-(-2x^3) = +2x^3$. \quad $x \cdot x^2 = x^3$. \quad $4x^2 \cdot x = 4x^3$. \quad $x^0 = 1$. \\
2. Alles sind $x^3$-Terme (bis auf die Zahl am Ende): $8 + 3 + 2 + 1 + 4 = 18$. \\
Ergebnis: $\mathbf{18x^3 - 1}$

\textbf{d)}
Brüche auflösen: $2x^3 + 3x^3 - 2x^3 + x^3 + x^3$. \\
Zusammenfassen: $2+3-2+1+1 = 5$. \\
Ergebnis: $\mathbf{5x^3}$

\textbf{e)}
Terme vereinfachen: \\
$2x^{-2} \cdot x^5 = 2x^3$ \\
$\frac{x^6}{x^3} = x^3$ \\
$(x^2)^2 \cdot x^{-1} = x^4 \cdot x^{-1} = x^3$ \\
$\frac{1}{x^{-3}} = x^3$ \\
Rechnung: $2x^3 + x^3 - x^3 + 3x^3 - x^3 = 4x^3$. \\
Ergebnis: $\mathbf{4x^3}$

\textbf{f)}
Terme ordnen: \\
$x^2y$-Terme: $3x^2y - 2x^2y - x^2y = 0$ (fällt weg!) \\
$xy^2$-Terme: $4xy^2 + 5xy^2 + 2xy^2 = 11xy^2$. \\
Ergebnis: $\mathbf{11xy^2}$

\textbf{g)}
Exponenten berechnen: $(a^2)^3=a^6$, $(a^3)^2=a^6$, $(a^2)^0=1$. \\
Rechnung: $a^6 \cdot a^4 \cdot a^6 : a^5 \cdot 1 \cdot a^1$. \\
Hochzahlen: $6 + 4 + 6 - 5 + 0 + 1 = 12$. \\
Ergebnis: $\mathbf{a^{12}}$

\textbf{h)}
Werte berechnen: \\
$5 \cdot 1 = 5$. \quad $1$. \quad $-1$. \quad $1$. \quad $5x + 5x = 10x$. \quad $4 - 4 = 0$. \\
Zahlen: $5 + 1 - 1 + 1 = 6$. \\
Ergebnis: $\mathbf{10x + 6}$

\textbf{i)}
Einzelteile: \\
$\frac{27a^3}{9a} = 3a^2$. \quad $-2a^2$. \quad $4a^2$. \quad $-a^2$. \quad $5a^2$. \\
Rechnung: $3a^2 - 2a^2 + 4a^2 - a^2 + 5a^2 = 9a^2$. \\
Ergebnis: $\mathbf{9a^2}$

\textbf{j)}
Vereinfachen: \\
Term 1: $2a^2b$. \\
Term 2: $3a^2b$. \\
Term 3: $-2a^2b$. \\
Term 4: $\frac{a^3b^3}{ab^2} = a^2b$. \\
Term 5: $-5a^2b$. \\
Term 6: $+ab$ (Passt nicht zu den anderen!). \\
$a^2b$-Terme: $2 + 3 - 2 + 1 - 5 = -1a^2b$. \\
Ergebnis: $\mathbf{-a^2b + ab}$

\end{document}